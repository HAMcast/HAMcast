\section{Introduction}

Many application in the Internet are based on data distribution between
one or more sender and many receivers, a scheme also known as group
communication. Instead of using many single links between each sender and
receiver pair, i.e. unicast, it is more efficient to use multicast.
Though there exists a wide range of different multicast technologies,
e.g. IP- or overlay-multicast (based on a P2P network), two things are
missing. First, there is no common multicast service API to develope
group applications independent of the underlying protocol. And second,
until now there is no universal multicast service available over the
Internet. It seams that both reasons are connected.

In this document we describe the system architecture of \hamcast\, first
in general (see section~\ref{sec:design}) and second the structure of its
prototype implementation (see section~\ref{sec:code}). We also give an
introduction for the usage of our prototype in section~\ref{sec:usage}
and how to develope group communication applications, that implement the
\hamcast-API. Finally we also present detailed insights in the source code
based on Doxygen in section~\ref{sec:doxygen}.
