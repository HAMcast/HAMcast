\section{Use it}
\label{sec:usage}

Hamcast is written in C++. Fortunately you can mix Objective-C, C and C++ in \textit{.mm} files. Rename the ending from \textit{.m} to \textit{.mm} if you want to use hamcast in a class.

The header file for hamcast is named \textit{hamcast.hpp} and should be included as shown below:

\begin{lstlisting}
	#include "hamcast/hamcast.hpp"
\end{lstlisting}

The multicast socket implementation of hamacst is called ``\textit{multicast\_socket}''. As expected the socket provides the functions \lstinline^send^, \lstinline^receive^, \lstinline^join^ and \lstinline^leave^. Hamcast uses ``\textit{URIs}'' as group identifiers, complient with RFC 3986. Both join and leave expect a URI as a parameter. The data you receive with the ``\textit{multicast\_socket}'' is packed in a ``\textit{multicast\_packet}'' and can be obtained via the \lstinline^data()^ and \lstinline^size()^ member functions.

Lets see how this works. The following code is an example function mixing Objective-C and C++. It creates a multicast\_socket, joins a group, interprets received data as strings and prints it on your Xcode console.

\newpage
\begin{lstlisting}
#include "hamcast/hamcast.hpp"
...
	
- (void) receive
{
    hamcast::multicast_socket sck;
	hamcast::multicast_packet message;
    hamcast::uri group("ip://239.255.0.1:30001");
    
    if(group.empty())
    {
        NSLog(@"Not a valid group!");
        return;
    }
    
    sck.join(group);
                       
    while(true)
    {
        message = sck.receive();
        if(message.empty())
        {
            std::cout << "received empty multicast_packet"
            		  << std::endl;
        }
        else
        {
        	std::string output(reinterpret_cast<const char*>
        		(message.data()), message.size());
        	std::cout << "received: " << output << std::endl;
        }
    }
}

...
\end{lstlisting}